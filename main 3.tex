\documentclass[letterpaper, 11pt]{article}
\usepackage{latexsym}
\usepackage{amssymb}
\usepackage{times}
%\usepackage[in]{fullpage}
\usepackage{amsmath,amsfonts,amsthm}
\usepackage{graphicx}
\usepackage{fancyhdr}
\usepackage{lastpage}
\usepackage{extramarks}
\usepackage[usenames,dvipsnames]{color}
\usepackage{amsmath}
\usepackage{amsthm}
\usepackage{amsfonts}
\usepackage{changepage}
\usepackage{lineno}
\usepackage[plain]{algorithm}
\usepackage{algpseudocode}
\usepackage{tikz}
\usepackage{hyperref}

%\documentclass[11pt]{article}
%\pagestyle{myheadings}
%\usepackage[ruled,nothing]{algorithm}
%\usepackage{algorithmic}
%\usepackage[dvips]{epsfig,graphicx}
%\numberwithin{equation}{section}

\bibliographystyle{plain}

\newenvironment{newalgo}[2]{\begin{algorithm}

\caption{\textsc{#1}}\label{#2}

\begin{algorithmic}[1]}{\end{algorithmic}\end{algorithm}}



\newcommand{\gm}{\gamma}
\newcommand{\wh}{\widehat}
\newcommand{\rep}{representation}
\newcommand{\rv}{random variable}
\newcommand{\la}{\lambda}
\newcommand{\wt}{\widetilde}
\newcommand{\st}{such that}
\newcommand{\slvary}{slowly varying}
\newcommand{\ma}{moving average}
\newcommand{\regvary}{regularly varying}
\newcommand{\asy}{asymptotic}
\newcommand{\ts}{time series}
\newcommand{\id}{infinitely divisible}
\newcommand{\seq}{sequence}
\newcommand{\fidi}{finite dimensional \ds}

\newcommand{\ble}{\begin{lemma}}
\newcommand{\ele}{\end{lemma}}
\newcommand{\bfX}{{\bf X}}
\newcommand{\pro}{probabilit}
\newcommand{\BX}{{\bf X}}
\newcommand{\BY}{{\bf Y}}
\newcommand{\BZ}{{\bf Z}}
\newcommand{\BV}{{\bf V}}
\newcommand{\BW}{{\bf W}}
\newcommand{\reals}{{\mathbb R}}
\newcommand{\bbr}{\reals}

\newcommand{\balpha}{\mbox{\boldmath$\alpha$}}
\newcommand{\bbeta}{\mbox{\boldmath$\beta$}}
\newcommand{\bmu}{\mbox{\boldmath$\mu$}}
\newcommand{\tbmu}{\mbox{\boldmath${\tilde \mu}$}}
\newcommand{\bEta}{\mbox{\boldmath$\eta$}}


\def \br#1{\left \{#1 \right \}}
\def \pr#1{\left (#1 \right)}

\newcommand{\Gm}{\Gamma}
\newcommand{\ep}{\epsilon}


\newtheorem{lemma}{Lemma}[section]
\newtheorem{figur}[lemma]{Figure}
\newtheorem{theorem}[lemma]{Theorem}
\newtheorem{proposition}[lemma]{Proposition}
\newtheorem{definition}[lemma]{Definition}
\newtheorem{corollary}[lemma]{Corollary}
\newtheorem{example}[lemma]{Example}
\newtheorem{exercise}[lemma]{Exercise}
\newtheorem{remark}[lemma]{Remark}
\newtheorem{fig}[lemma]{Figure}
\newtheorem{tab}[lemma]{Table}
\newtheorem{fact}[lemma]{Fact}
\newtheorem{test}{Lemma}

\newcommand{\play}{\displaystyle}

\newcommand{\ms}{measure}
\newcommand{\beao}{\begin{eqnarray*}}
\newcommand{\eeao}{\end{eqnarray*}\noindent}
\newcommand{\beam}{\begin{eqnarray}}
\newcommand{\eeam}{\end{eqnarray}\noindent}

\newcommand{\halmos}{\hfill\mbox{\qed}\\}
\newcommand{\fct}{function}
\newcommand{\ins}{insurance}
\newcommand{\ds}{distribution}

\newcommand{\one}{{\bf 1}}
\newcommand{\eid}{\buildrel{\rm d}\over {=}}
\newcommand {\Or}{\rm ORDER}
\newcommand {\In}{\rm INTER}

\newcommand{\bbd}{{\mathbb D}}
\newcommand{\vi}{$V_{ij}$ }
\newcommand{\rr}{R^{\prime\prime}}
%\newcommand{\R}{R^\prime}
\newcommand{\ci}{\frac{1}{c}}
\newcommand{\Vi}{V(n)}
\newcommand{\dR}{\mathcal R}
\newcommand{\md}[1]{\left(\ \rm{mod}\ \it{#1}\right)}
\newcommand{\So}{s}
%\begin{document}
%\def\DoubleSpace{\baselineskip=24pt}
%\DoubleSpace \sloppy

\begin{document}



\title{Homework \#8 \\ Algorithms I \\ 600.463 \\Spring 2017}
\author{\textbf{Due on:} Thursday, April 27th, 11:59pm \\
\textbf{Late submissions:} will NOT be accepted\\
\textbf{Format:} Please start each problem on a new page.
\\\textbf{Where to submit:} On Gradescope, under HW8.
\\ Please type your answers; handwritten assignments will not be accepted.
\\ To get full credit, your answers must be explained clearly,\\ with enough details
and rigorous proofs.
\\}

\maketitle

%%%%%%%%%%%%%%%%%%%

\section{Problem 1 (20 points)}

A bipartite graph $G=(V,E)$, where $V = L \cup R$, is \textit{d-regular} if every vertex $v \in V$ has degree exactly $d$.  
\begin{itemize}
\item[\textbf{(a)}] Show that for every \textit{d-regular} bipartite graph, $|L|=|R|$.
\item[\textbf{(b)}] Model the maximum \textit{d-regular} bipartite matching as a max-flow problem. Show that the max-flow value from $s$ to $t$ in the formulation is $|L|$.
\item[\textbf{(c)}] Prove that every \textit{d-regular} bipartite graph has a matching of cardinality $|L|$.

\end{itemize}
\newpage
\section*{Problem 1 Answer}
\begin{itemize}
\item[\textbf{(a)}] Show that for every \textit{d-regular} bipartite graph, $|L|=|R|$.
\newline \newline \textbf{Solution:} The proof can be simply described in the following manner.T he definition of d-regular just means that for each vertex in the L, there will be $d$ edges connecting to a vertex in R. For every edge in the graph, connect the node in R with a node in L. Then, the total number of edges is $|R|$*$d$ and $|L|$*$d$, and therefore by closure $|L| = |R|$.

\item[\textbf{(b)}] Model the maximum \textit{d-regular} bipartite matching as a max-flow problem. Show that the max-flow value from $s$ to $t$ in the formulation is $|L|$.

\newline \newline \textbf{Solution:} In CLRS, chapter 26, particularly in the the figure 26.3, a bipartite matching problem is reduced to a network flow and this is the exact idea that we need to use to model the d-regular bipartite matching as a max-flow problem. In the figure, every edge in the graph has unit capacity, and so we can derive a function where there is a unit flow out of every edge from $s$ and every edge into $t$. As all edges will have unit capacity, we can assume further that all edges have flow 1, from s in L and all vertices in R to t. Since every edge from $s$ is accounted for and used in the flow, then this is a maximum flow and therefore the max-flow value from $s$ to $t$ is indeed $|L|$.

\item[\textbf{(c)}] Prove that every \textit{d-regular} bipartite graph has a matching of cardinality $|L|$.

\newline \newline \textbf{Solution:} In CLRS Chapter 26, we get a theorem specifically Theorem 26.11 which states the Integral Theorem. If the capacity function c takes only integral values, then the maximum flow f produced by the Ford-Fulkerson method has the properpty that $|f|$ is integer-valued. For all vertices $u$ and $v$ the value $f(u,v)$ of the flow is an integer. And so since every edge in the max-flow values in the formulation showed above has an integral capacity, then there must be an integral flow with the value $|L|$ as described in the theorem. And so every $d$-regular bipartite graph has a matching of cardinality $L$.


\end{itemize}


\newpage
\section{Problem 2 (20 points)}
In the \textit{maximum k-cut} problem, we are given an undirected graph $G=(V,E)$, and non-negative weights $w_{ij} \ge 0$ for all $(i,j) \in E$. The goal is to partition the vertex set $V$ into $k$ parts $V_1, \dots, V_k$ so as to maximize the weight of all edges whose endpoints are in different parts (i.e., $\text{max}_{(i,j)\in E:i \in V_a, j\in V_b, a\ne b}w_{ij}$).\\
Give a randomized $\frac{k-1}{k}$-approximation algorithm for the MAX $k$-CUT problem.\\
Hint: please review Chapter 5.1 of the book---``The Design of Approximation Algorithms'' (Williamson and Shmoys 2010), and try to solve this problem using similar ideas.

\newpage

\section*{Problem 2 Answer}
Here is an algorithm that partitions the vertex set $V$ into $k$ parts and maximizes the weight of all edges whose end-points are in different parts.  

\begin {algorithm}[]
    \begin{algorithmic}[1]
        \Function{Random-Approximation}{}
        \ForAll{$V_1, \dots, V_k$}
        \State{}set to $null$
        \EndFor
        \ForAll{$i$ $\leftarrow{}$  1 to $n$}
        \State{}uniform selection $x$ from $[0,1]$
        \State{} \textbf{if} x \in $ [j/k, (j + 1)/k]$}
        \newline\indent\State{}{$V_{j+1} \leftarrow V_{j+1} \cup {i}$}
        \EndFor
        \EndFunction{}
    \end{algorithmic}
\end{algorithm}

\newline
The algorithm is supposed to produce as many partitions as there are $k$. In layments term, the goal is to have a partitioning of  [0, 1) into the propoer subranges where there is also one subset per subrange. Let's note the following: (i.e., $\text{max}_{(i,j)\in E:i \in V_a, j\in V_b, a\ne b}w_{ij}$).\\
\newline
The entire function for assignment of the variables to the randomized uniform selection of $x$ will run in polynomial time, where the random variable is either 1 or 0, or in other words, when there is an edge connecting vertices in the different sets the variable will be set to 1 and 0 in all other cases. So, an edge can either connect vertices in the same subset or connect vertices in different subsets. It will be 1 if for some $a,b, i$ $\in$ $V_a$ and $j \in V_b$ and a$!=$b.Here is an approximation analysis proof (where $W$ is the random variable) that proves mathematically what the end goal is that I have described above with much more precision and detail. This link was found accompanied with a Google search result of Williamson and Shmoys 2010 Chapter 5.1 (courses.csail.mit.edu/6.891-s00/pss2.ps). Ultimately, we can find that the summation of $(i,j)$ $\in$ $E$ of $w_{i*j}$ multiplied by $(k-1)/k$ and thus simplified one step further that $\frac{k-1}{k}$-approximation algorithm works for this MAX $k$-CUT problem





\end{document}

%%%%%%%%%%%%%%%%%%%%%%%%%%%%%
